\documentclass{article}
\usepackage[utf8]{inputenc}

\title{Вероятностное программирование}
\author{Ilya Noskov }
\date{March 2020}

\usepackage{natbib}
\usepackage{graphicx}
\usepackage[russian]{babel}
\begin{document}

\maketitle

\section{Введение}
Вероятностное программирование - парадигма программирования в которой используются вероятностные модели, выводы из которых получаются автоматическим методом. Эта парадигма представляет собой попытку объеденить вероятностное моделирование и традиционные языки программирования общего назначения, для того что бы сделать вероятностное моделирование более широко и проще использующимся.
\section{Языки программирования и фреймворки}
В текущий момент времени большинство языков вероятностного программирования выполнены в виде фреймворков на уже существующих языках программирования. 
\begin{itemize}
  \item Anglician(расширение для Clojure)
  \item WebPPL (расширение JavaScript)
  \item Stan (платформа с интерфейсами к языкам R, Python, MATLAB, Julia, Stata)
  \item PyMC, Pyro(базируется на PyTorch), Edward (пакеты для Python)
  \item Figaro и FACTORIE (Библиотеки для языка Scala)
  \item ProbLog (расширение языка Prolog)
\end{itemize}
Каждый из языков отличается от остальных и имеет ряд преимуществ перед другими, ряд более подходящих конкретно для него задач. Например Problog отличается своим подходом т.к. он базируется на языке логического программирования. Языки могут отличатся методами статичстического вывода который они используют, типов переменных, дискретных или непрерывных. Или больше подходить под определенный класс задач, так например FACTORIE больше используется при работе с NLP. 
\section{Основная часть}
Грубо говоря, программа вероятностного программирования состоит из двух частей, это вероятностная модель и алгоритм вывода. При этом вероятностное программирование используется для так называемого вероятностного обоснования. Вероятностное обоснование это подход к принятию решений когда факторы влияющие на возможный исход недетерминированны. Ранее, вероятностные модели были не так широко применимы из за того фактора что для их подсчета в ручную было необходимо слишком много времени и ресурсов, тогда как эта проблема решалась при появлении современных ЭВМ. Вероятностная парадигма используется для решения трёх классов задач:
\\*
1)Предсказывать будущий исход.
\\*
2)Сделать вывод об ходе событий.
\\*
3)Обучаться на предыдущих предсказаниях для того что бы лучше предсказывать в будующем.
\\*
Данный подход позволяет использовать вероятностное программирование в таких задачах как обнаружение кибер угроз, при создании рекомендательных систем, систем компьютерного зрения, в эпидемологии. 
Например вероятностное программирование может решать задачу анализа возможной эпидемии. Модель для данной Параметрами для данной задачи выступали бы коэффицент r0, показатель популяционного иммунитета, количество уже заболевших людей. Вероятностное программирование бы могло ответить на вопрос того что эпидемия будет развиваться, или пойдет на спад.Модель могла бы показать влияние каждого из параметров на вывод, например насколько важен популяционный иммунитет для успешного предотвращения развития эпидемии. Так же вероятностная модель отличается гибкостью и возможностью изменяться в зависимости от предудщих своих предсказаний, их точности. Например, если были полученны обновленные данные про коэфицент r0, то вероятности тех или иных выводов изменяются.
Так же, на текущем этапе вероятностная парадигма вместе с глубоким обучение образуют новый подход к искусственному интеллекту, который называется глубокое вероятностное обучение. Этот подход стремится к тому что бы совместить преимущества глубокого обучения с наилучшими сторонами вероятностного программирования. Преимуществом этого подхода относительно классического глубокого обучения является необходимость в меньшем количестве обучающих данных и идентичным временем необходимым для обучения. Так же, данный подход позволяет упростить и построенние самой модели для обучения, её размеров.
\end{document}
